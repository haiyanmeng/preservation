\section{Related Work}

Generally, there are three approaches to preserve software environment:
hardware preservation, migration and emulation.  Hardware
preservation preserves the original software and its original operating
environment. 
Software migration technique~\cite{cifuentes1996binary,mancl2001refactoring} was used to facilitate running software on new machines.
However, migration often involves the re-compiling and re-configuring
the source code to accommodate a new hardware platform and software environment.
Emulation recreates the original software and hardware environment by
programming future platforms and OSs. One common solution to implement this is
virtual machine. According to the usage and emulation degree of the real
machine, virtual machine can be divided into system virtual machine and process
virtual machine. 
The working principle, design principle and
performance evaluation of system virtual machine were illustrated in~\cite{goldberg1974survey, smith2005architecture}. 
The
functionality of system VM to support different guest operating systems was illustrated in~\cite{barham2003xen,kivity2007kvm,rosenblum1999vmware}.
F. Esquembre~\cite{esquembre2004easy} illustrated how JVM, one process virtual machine, can expedite the creation of
scientific simulations in Java. 
The pros and cons of these three approaches were discussed in~\cite{matthews2009towards,phelps2005no,hong2010software}.

The preservation of computing environment and software environment was treated as one entirety in~\cite{matthews2009towards,phelps2005no,hong2010software}. However, frequently changing experiment software makes the maintenance of the preserved experimental environment very complex. 
CernVM~\cite{buncic2010cernvm} treated them as two different categories. The preservation of computing environment is implemented with CernVM, and the preservation of software environment is based on a CernVM filesystem(CVMFS) specifically designed for efficient software distribution.

The importance of preserving software in source code format was emphasized in~\cite{zabolitzky2002preserving,castagne2013consider}. 
However, CVMFS~\cite{buncic2010cernvm} published pre-built and configured experiment software releases to avoid the time-consuming software building procedure. 

Attempts from different perspectives to facilitate the reproduction of scientific experiments utilizing a preserved software library have been made. 
The software distribution mechanism over network was discussed in~\cite{compostella2010cdf, blomer2011cernvm}.
J. R. Rice et al.~\cite{rice1996scientific} made the reproduction process easier through the integration of user interface, scientific software libraries, knowledge base into problem-solving environment.
S. R. Kohn et al.~\cite{kohn2001divorcing} tried to enable the creation and distribution of language-independent software library by addressing language interoperability.
a scalable, distributed and dynamic workflow system for digitization processes was proposed in~\cite{schoneberg2013scalable}.
A distributed archival network was designed in~\cite{subotic2013distributed} to facilitate process-oriented automatic long-term digital preservation.
M. Agosti et al.~\cite{agosti2012envisage} aimed to help non-domain users to utilize the digital archive system developed for domain experts.

Current mechanisms of preserving scientific experiments assume that all the data and software mentioned in the experiments are necessary for the reproduction of the experiments. However, this is not always right. In some cases, the original author may leave additional code referring to irrelative data and software in the program. One mechanism, which can figure out the absolutely relevant data and software of one experiment, is important for both the preservation and reproduction of scientific experiments.

B. Matthews et al.~\cite{matthews2008significant} introduced one conceptual framework for software preservation from several case studies of software preservation.
One tool to capture software preservation properties within a software environment was designed in~\cite{matthews2010framework} through a series of case studies conducted to evaluate the software preservation framework.
L. R. Johnston et al.~\cite{johnston2014workflow} proposed one overall data curation workflow for 3-5 case studies of preserving research data.
Two case studies~\cite{borgman2012data} were conducted to figure out the properties of data to be reused in the future.

