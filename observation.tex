\section{Challenges and Observations in Reproducing HEP Applications}

The \emph{TauRoast} and \emph{Athena} application specification was provided to us in the form of an
email which described, in prose, how to obtain the source,
build the program, and run it correctly on one specific
machine at our home institution, with no particular guarantee that
it will run anywhere else in the world. Such level of documentation and exchange of information about a software is typically routine in the scientific world. 
We provide the challenges we faced in capturing the application details in a reproducible form and then preserving it for subsequent reuse:

%\comment{Although this starting point may seem extreme, it is
%perfectly natural for collaborators to share configurations
%with each other in this form, and to rely on the presence
%of a working environment with appropriate dependencies already
%installed.  From this starting point, the authors played the
%role of curators, whose job it is to prepare the application
%for permanent archival.
%
%First, we elaborated the email instructions into an
%executable script that obtains the dependencies and then
%executes the analysis.  The script declares the necessary
%environment variables, downloads and checks out the necessary source code,
%builds it appropriately, calls initialization scripts in
%the dependent software, and then runs the analysis.
%A few rounds of correction with the original author were necessary
%to obtain all the dependencies and run the artifact correctly.
%(The original instructions introduced in the email also indicated how to run the application
%within a production batch system.  For the purposes of preservation,
%we consider the execution infrastructure to be distinct from the application,
%and leave it out of consideration for now.)
%
%The process of elaborating the program into a script revealed
%several observations about this type of application:}

\begin{itemize}

\item {\bf Many Explicit External Dependencies.}  Due to the distributed nature of HEP applications, the applications typically depend on a large number of networked external resources. 
Capturing these dependencies is straightforward if they are static in nature and the resources can be accessed without secure cryptographic functions. 
For instance, in TauRaoast some source code external dependencies can be obtained by simply downloading the specific libraries from a web page. 
However, other static dependencies require making secure connections to Github resources, and CVS servers for downloading source. 
While TauRoast is completely local in its execution, dynamic remote dependencies that arise in a distributed execution require auditing dependencies on all machines where execution takes place. 

%\begin{TM} 
%each with a different access method and data source.
%While we knew in advance that it depended upon the large CMSSW distribution,
%it was not apparent until elaborating the script that it depended upon
%two different Github repositories for the Tau source,
%a CVS server at CERN for some configuration information, a public web page
%for the PyYAML library, and the public home page of a Notre Dame student
%for one missing header file.  (The latter is particularly troubling!)
%While, at some level, the authors and users of these software know of these dependencies, they are often missing in
%informal communications or forgotten once the dependency is installed.
%However, once known, they are at least expressed explicitly within the script.
%\end{TM}

\item {\bf Many Implicit Local Dependencies.} The application data and code is distributed on five networked filesystems, which are 
typically mounted on a particular machine that is made accessible to the application developer. In TauRoast the a large amount of data to be analyzed was stored on an HDFS~\cite{borthakur2008hdfs} cluster,
some configuration data was stored on a CVMFS~\cite{blomer2011cernvm} filesystem,
and a variety of software tools were on an NFS~\cite{howard1988scale},
PanFS~\cite{welch2008scalable} and AFS~\cite{sandberg1985design} systems. In \emph{Athena} the data and code were available through CVMFS filesystem. 
Since these filesystems appear local to the application machine, it is important to check and capture mounted filesystems and their respective mount points. 

%\begin{TM}A much harder problem is that the
%application assumed the presence of many different components in the local
%filesystem view. It would be tempting to capture all of these by simply
%storing a virtual machine image containing the local filesystem. However,
%the application depended on no less than {\bf five} networked filesystems
%available on a particular machine available to the author:
%the data to be analyzed was stored on an HDFS~\cite{borthakur2008hdfs} cluster,
%some configuration data was stored on a CVMFS~\cite{blomer2011cernvm} filesystem,
%and a variety of software tools were on an NFS~\cite{howard1988scale},
%PanFS~\cite{welch2008scalable} and AFS~\cite{sandberg1985design} systems.
%The original authors were not aware of many of these dependencies,
%because they simply relied on local administrators to configure the
%software and make it available.\end{TM}

\item {\bf Configuration Complexity.} To correctly reproduce an application implies that run-time configurations and consistency checks on the available software are effective captured and preserved. 
For CMS,  {\tt scram} software management tool, is used to locate
the appropriate version of software,  set environment variables such as the PATH, run any
tool-specific configuration, and do the same for all software on which it depends. A reproducible framework must capture the working of such software management tools so that they can conduct similar
checks on a new machine. 

%\begin{TM}As a means of controlling the complexity
%of dependent software packages, the high energy physics community has developed
%a number of tools that perform run-time configuration and consistency checks
%of the available software. {\tt scram} is the software management tool used
%by the CMS experiment.  Before running any code, {\tt scram} is used to locate
%the appropriate version of software,  set environment variables such as the PATH, run any
%tool-specific configuration, and do the same for all software on which it depends.
%If the correct versions are not available, {\tt scram} halts and emits an error.
%While this procedure has great value for consistency, it also introduces a significant cost
%because it involves a large number of nested scripts traversing a filesystem,
%repeatedly looking up metadata.  In our example, the time to perform this configuration
%with a cold cache is about 14 minutes, which is almost as large as the actual analysis
%run, which takes 20 minutes.\end{TM}

\item {\bf High Selectivity.}  
Although the total size of the resources accessed by HEP
programs is very large, the size of the data and software actually used are much smaller.
Often, an entire repository or data source is named within the script, but the program
only needs a handful of items from that source.  For example, the data is stored on an
HDFS filesystem with 11.6TB of data, but only 20GB are actually consumed by the program. 
The reason for this great reduction is at first each BEAN event contains a large amount of information, and the TauAnalysis throws away a lot of the irrelevant event information, keeping only the relevant bits.
The CMSSW repository is 88.1GB in total
but only 448.3MB of source is checked out, and the actually used software only
measured 6.3MB.  In a few cases, a source of software is named but never actually accessed.
(For example, our original script includes the Open Science Grid software stack in the PATH, but does not actually use it.)
%We suspect that end users are accustomed to missing dependencies and thus get in the habit of adding commonly used software,
%whether it is needed or not. 
Thus there is a size tradeoff of capturing dependencies mentioned in the program and dependencies actually used in the program.
A reproducible framework must include robust rules about not including superfluous dependencies, but including unused dependencies that may potentially get used during program execution.  

\item {\bf Rapid Changes in Dependencies.}  Over the course of three months
between collecting the initial email, analyzing the program, and writing this
paper, the computing environment continuously changed.  The CMSSW software
distribution released a new version, the target execution environment was upgraded
to a new operating system, and the application deprecated the use of CVS for obtaining
the software. In Athena, the computing environment can potentially change daily, since upgrades to the software framework occur on a nightly basis.  
While the users of this software seem be accustomed to constant change,
any preservation technique will have to be very cautious about relying upon an
external service, even one that may appear to be highly stable.

\item {\bf Additional Dependencies.} Capturing the necessary and sufficient dependencies that are part of an application is sufficient for repeatability, but possibly for not reproducibility, where is different additional modules
must also be preserved so that the resulting application can be extended properly. In Athena framework, for instance, the software library needed to include basic software programs such as make, awk, sed that can aid further development. 
\end{itemize}

\if 0

\begin{table}
    \centering
    \begin{tabular}{|l|}
        \hline
        {\bf Version 1: Email}\\ \hline
        1. Create a CMS release,\\
            \hspace{9pt} e.g. {\tt cmsrel CMSSW\_5\_3\_11\_patch3} \\
        2. Install the BEAN packages as the instructions: \\
            \hspace{9pt} {\tt \url{https://github.com/cms-ttH/BEAN/blob/...}}\\
        3. Install grid-control: \\ 
            \hspace{9pt} {\tt svn co \url{https://ekptrac.physik.uni-ka/...}} \\
        4. INstall the TauAnalysis package: \\
           \hspace{9pt} {\tt git clone \url{https://github.com/matz-e/...}} \\
           \hspace{9pt} {\tt scram b -j32} \\
        5. Fix grid\_control.cfg and run it. \\
        6. Perform the actual tau roast program. \\ 
        \hline
        {\bf Version 2: Script}\\ \hline
        {\tt setenv CMSSW\_BASE CMSSW\_5\_3\_11\_patch3} \\
        {\tt cmsrel \$HOME/\$CMSSW\_BASE} \\
        {\tt cvs co -r V03-09-23 PhysicsTools/PatUtils} \\
        {\tt git clone \url{https://github.com/cms-ttH/BEAN.git}} \\
        {\tt wget -r \url{http://nd.edu/~abrinke1/...}} \\
        {\tt scram b -j32} \\
        {\tt wget \url{http://pyyaml.org/download/pyyaml/PyYAML...}}\\
        \#the experiment data is from HDFS \\
        {\tt cd \$HOME/\$CMSSW\_BASE/src/PyYAML-3.10}\\
        {\tt cmsenv}\\
        {\tt python setup.py install --user} \\
        {\tt scripts/roaster data/generic\_ttl.yaml} \\ 
        \hline
        {\bf version 3: Formulated Script} \\ \hline
        {\tt setenv CMSSW\_BASE CMSSW\_5\_3\_11\_patch3} \\
        {\tt setenv {\bf GIT} \url{https://github.com}} \\
        {\tt setenv {\bf PYYAML} \url{http://pyyaml.org}} \\
        {\tt setenv {\bf ND} \url{http://nd.edu}} \\
        {\tt cmsrel \$HOME/\$CMSSW\_BASE} \\
        {\tt cvs co -r V03-09-23 PhysicsTools/PatUtils} \\
        {\tt git clone \${\bf GIT}/cms-ttH/BEAN.git} \\
        {\tt wget -r \${\bf ND}\url{/~abrinke1/ElectronEffectiveArea.h}} \\
        {\tt scram b -j32} \\
        {\tt wget \${\bf PYYAML}/download/pyyaml/PyYAML...}\\
        \#the experiment data is from HDFS \\
        {\tt cd \$HOME/\$CMSSW\_BASE/src/PyYAML-3.10}\\
        {\tt cmsenv}\\
        {\tt python setup.py install --user} \\
        {\tt scripts/roaster data/generic\_ttl.yaml} \\ 
        \hline
       {\bf Version 4: Fine-Grained Toolkit - Package}\\ \hline
        {\tt setenv CMSSW\_BASE CMSSW\_5\_3\_11\_patch3} \\
        {\tt setenv {\bf GIT} \url{https://github.com}} \\
        {\tt setenv {\bf PYYAML} \url{http://pyyaml.org}} \\
        {\tt setenv {\bf ND} \url{http://nd.edu}} \\
        {\tt cmsrel \$HOME/\$CMSSW\_BASE} \\
        {\tt cvs co -r V03-09-23 PhysicsTools/PatUtils} \\
        {\tt git clone \${\bf GIT}/cms-ttH/BEAN.git} \\
        {\tt wget -r \${\bf ND}\url{/~abrinke1/ElectronEffectiveArea.h}} \\
        {\tt scram b -j32} \\
        {\tt wget \${\bf PYYAML}/download/pyyaml/PyYAML...}\\
        \#the experiment data is from HDFS \\
        {\tt cd \$HOME/\$CMSSW\_BASE/src/PyYAML-3.10}\\
        {\tt cmsenv}\\
        {\tt python setup.py install --user} \\
        {\tt scripts/roaster data/generic\_ttl.yaml} \\ 
        \hline
    \end{tabular}
    \caption{Scripts of each Solution}
    \label{table:scripts}
\end{table}

\fi

